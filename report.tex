\documentclass[a4paper]{article}

\usepackage[utf8]{inputenc}
\usepackage[french]{babel}
\usepackage{amsmath}
\usepackage{xcolor}
\usepackage{graphicx}
\usepackage{listings}
\usepackage{caption}

\graphicspath{{./slides/img/}}
\DeclareGraphicsExtensions{.png, .jpeg, .jpg, .svg}

\title{Faster Text Fingerprinting}
\author{Hugo \textsc{Mougard} et Soufian \textsc{Salim}}
\date{\today}

\begin{document}

\maketitle

\tableofcontents

\section{Introduction}

\subsection{Qu'est ce qu'un \textit{fingerprint} ?}

Pour $s = s_{1} .. s_{n}$ un texte sur un alphabet $\Sigma$, un \textit{fingerprint} $f$ est un ensemble des caractères distincts contenus dans l'une de ses sous-chaînes. Le \textit{fingerprinting} consiste à calculer l'ensemble $\mathcal{F}$ de tous les \textit{fingerprints} $f$ de toutes les sous-chaines de $s$.  \newline

Par exemple, pour la chaîne $s = a_{1} b_{2} c_{3} a_{4} a_{5} d_{6} d_{7} b_{8}$, le \textit{fingerprint} de $[3,7]$ est $cad$.

\subsection{Applications possibles}

\subsection{Problèmes soulevés}

\subsection{Portée de l'article}

\subsection{Avant de commencer}

\subsubsection{Localisations maximales}

On note $\mathcal{C}$ un ensemble de lettres de $\Sigma$. Une location maximale de $\mathcal{C}$ dans $s$ est un intervalle $[i,j]$ tel que :
		
\begin{itemize}
	\item $\mathcal{C}_{s}(i,j) = \mathcal{C}$
	\item Si $i > 1$, $s_{i-1} \notin$ $\mathcal{C}_{s}(i,j)$
	\item Si $j < n$, $s_{j+1} \notin$ $\mathcal{C}_{s}(i,j)$
\end{itemize}

On note $\mathcal{L}$ l'ensemble des locations maximales de tous les \textit{fingerprints} de $\mathcal{F}$. \newline

Par exemple, pour la chaîne $s = a_{1} b_{2} c_{3} a_{4} a_{5} d_{6} d_{7} b_{8}$, la seule location maximale de $abc$ est $\langle1,5\rangle$.

\subsubsection{Copies et classes d'équivalence}

Deux locations maximales $\langle i,j \rangle$ et $\langle k,l \rangle$ de $s$ sont des copies si $s_{i}..s_{j} = s_{k}..s_{l}$. On note $\mathcal{L}_{\mathcal{C}}$ l'ensemble des classes d'équivalence. \newline

Par exemple, pour la chaîne $s = a_{1} b_{2} a_{3} d_{4} a_{5} b_{6} a_{7}$, $\langle1,3\rangle$ et $\langle5,7\rangle$ sont des copies.

\subsection{Contexte}

\subsection{Apports de l'article}

\subsection{Vue d'ensemble}
	
\begin{enumerate}
	\item Construire l'arbre des suffixes de $s$
	\item À partir de l'arbre des suffixes, construire un arbre des participations
	\item À partir de l'arbre des participations, nommer tous les \textit{fingerprints} de $s$
\end{enumerate}

\section{\textit{Participation Tree}}

\subsection{Principes}

\subsection{Construction de l'arbre}

\subsubsection{Étape 1}

\subsubsection{Étape 2}

\subsubsection{Étape 3}

\subsubsection{Étape 4}

\subsubsection{Étape 5}

\subsection{Théorème}

\subsection{Du \textit{Suffix} Tree au \textit{Participation Tree}}

\subsubsection{Idées}

\subsubsection{Algorithme}

\subsubsection{Calcul de la participation de l'arète}

\subsubsection{Astuces algorithmiques}

\section{Nommage des \textit{fingerprints}}

\subsection{Pourquoi donner un nom aux \textit{fingerprints} ?}

\subsection{Nommage hiérarchique}

\subsection{Utiliser les deltas de participation entre \textit{fingerprints}}

\subsection{Nommage d'une liste de listes de participations}

\subsection{Nommage sur l'arbre}

\section{Conclusion}

\subsection{Solutions aux problèmes soulevés en introduction}

\subsection{Calcul de $\mathcal{F}$}

\end{document}