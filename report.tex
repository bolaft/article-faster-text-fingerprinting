\documentclass[a4paper,10pt]{article}

\usepackage{libertine}
\usepackage[utf8]{inputenc}
\usepackage[french]{babel}
\usepackage{amsmath}
\usepackage{xcolor}
\usepackage{graphicx}
\usepackage{listings}
\usepackage{caption}

\graphicspath{{./slides/img/}}
\DeclareGraphicsExtensions{.png, .jpeg, .jpg, .svg}

\lstset{showspaces=false}
\lstset{showtabs=false}
\lstset{extendedchars=true}
\lstset{columns=flexible}
\lstset{keepspaces=true}
\lstset{numbers=left, numberstyle=\tiny, stepnumber=1, numbersep=5pt}

\title{Faster Text Fingerprinting}
\author{Hugo \textsc{Mougard} et Soufian \textsc{Salim}}
\date{\today}

\begin{document}

\maketitle

\tableofcontents

\section{Introduction}

\subsection{Qu'est ce qu'un \textit{fingerprint} ?}

\subsection{Applications possibles}

\subsection{Problèmes soulevés}

\subsection{Portée de l'article}

\subsection{Avant de commencer}

\subsubsection{Localisations maximales}

\subsubsection{Copies et classes d'équivalence}

\subsection{Contexte}

\subsection{Apports de l'article}

\subsection{Vue d'ensemble}

\section{\textit{Participation Tree}}

\subsection{Principes}

\subsection{Construction de l'arbre}

\subsubsection{Étape 1}

\subsubsection{Étape 2}

\subsubsection{Étape 3}

\subsubsection{Étape 4}

\subsubsection{Étape 5}

\subsection{Théorème}

\subsection{Du \textit{Suffix} Tree au \textit{Participation Tree}}

\subsubsection{Idées}

\subsubsection{Algorithme}

\subsubsection{Calcul de la participation de l'arète}

\subsubsection{Astuces algorithmiques}

\section{Nommage des \textit{fingerprints}}

\subsection{Pourquoi donner un nom aux \textit{fingerprints} ?}

\subsection{Nommage hiérarchique}

\subsection{Utiliser les deltas de participation entre \textit{fingerprints}}

\subsection{Nommage d'une liste de listes de participations}

\subsection{Nommage sur l'arbre}

\section{Conclusion}

\subsection{Solutions aux problèmes soulevés en introduction}

\subsection{Calcul de $\mathcal{F}$}

\end{document}