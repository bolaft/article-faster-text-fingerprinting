\documentclass[10pt]{beamer}

\usetheme{Boadilla}
\beamertemplatenavigationsymbolsempty

\usepackage[utf8]{inputenc}
\usepackage[T1]{fontenc}
\usepackage[francais]{babel}
\usepackage[autolanguage]{numprint}
\renewcommand*{\rmdefault}{fxb}
\usepackage{listings}
\usepackage{xcolor}
\lstset{keywordstyle=\color{blue}, stringstyle=\color{green}}
\usepackage{amsmath}
\usepackage{wrapfig}
\usepackage{euler}
\usepackage{multicol}
\usepackage{graphicx}
\usepackage{tikz}
\usepackage{amsmath}
\usepackage{amssymb}

\usetikzlibrary{mindmap,trees}
\graphicspath{{./img/}}
\DeclareGraphicsExtensions{.png, .jpeg, .jpg}

\lstset{basicstyle=\footnotesize}

\title{Faster Text Fingerprinting}
\author{Hugo Mougard \& Soufian Salim}
\date{\today}

\begin{document}

\begin{frame}
  \maketitle
\end{frame}

\begin{frame}
	\tableofcontents
\end{frame}

\section{Introduction}

\begin{frame}
	\frametitle{Qu'est ce que le \textit{fingerprinting} ?}
	
	\begin{block}{Définition}
		Pour $s = s_{1} .. s_{n}$ un texte sur un alphabet $\Sigma$, un \textit{fingerprint} $f$ est un ensemble des caractères distincts contenus dans l'une de ses sous-chaînes. \newline
	
	Le \textit{fingerprinting} consiste à calculer l'ensemble $\mathcal{F}$ de tous les \textit{fingerprints} $f$ de toutes les sous-chaines de $s$. 
	\end{block}
	
	\begin{block}{Exemple}
		Pour la chaîne $s = a_{1} b_{2} c_{3} a_{4} a_{5} d_{6} d_{7} b_{8}$, le \textit{fingerprint} de $\langle 3,7 \rangle$ est $cad$.
	\end{block}
	
	Cet article s'intéresse au problème algorithmique du calcul de l'ensemble $\mathcal{F}$ dans $s$.	
\end{frame}

\section{Participation Tree}

\begin{frame}
	\frametitle{Travaux sur la complexité de l'algorithme}
	
	A. Amir, A. Apostolico, G. M. Landau et G. Satta, \textit{Efficient text fingerprinting via parikh mapping}, 2003 : 
	$\mathcal{O}(n|\Sigma|$ $log$ $n$ $log$ $|\Sigma|)$. \newline
	
	$\mathcal{O}(n$ $+$ $|\mathcal{L}|$ $log$ $|\Sigma|)$
	$\mathcal{O}(n$ $+$ $|\mathcal{L}_{\mathcal{C}})$ $log$ $|\Sigma|)$

\end{frame}

\section{Fingerprint Naming}

\section{Conclusion}

\end{document}