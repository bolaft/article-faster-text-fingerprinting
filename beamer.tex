\documentclass[10pt]{beamer}

\usetheme{Boadilla}
\beamertemplatenavigationsymbolsempty

\usepackage[utf8]{inputenc}
\usepackage[T1]{fontenc}
\usepackage[francais]{babel}
\usepackage[autolanguage]{numprint}
\renewcommand*{\rmdefault}{fxb}
\usepackage{listings}
\usepackage{xcolor}
\lstset{keywordstyle=\color{blue}, stringstyle=\color{green}}
\usepackage{amsmath}
\usepackage{wrapfig}
\usepackage{euler}
\usepackage{multicol}
\usepackage{graphicx}
\usepackage{tikz}
\usepackage{amsmath}
\usepackage{amssymb}

\usetikzlibrary{mindmap,trees}
\graphicspath{{./img/}}
\DeclareGraphicsExtensions{.png, .jpeg, .jpg}

\lstset{basicstyle=\footnotesize}

\title{Faster Text Fingerprinting}
\author{Hugo Mougard \& Soufian Salim}
\date{\today}

\begin{document}

\begin{frame}
  	\maketitle
\end{frame}

\begin{frame}
	\tableofcontents
\end{frame}

\section{Introduction}

\begin{frame}
	\frametitle{Qu'est ce que le \textit{fingerprinting} ?}
	
	\begin{block}{Définition}
		Pour $s = s_{1} .. s_{n}$ un texte sur un alphabet $\Sigma$, un \textit{fingerprint} $f$ est un ensemble des caractères distincts contenus dans l'une de ses sous-chaînes. \newline
	
	Le \textit{fingerprinting} consiste à calculer l'ensemble $\mathcal{F}$ de tous les \textit{fingerprints} $f$ de toutes les sous-chaines de $s$. 
	\end{block}
	
	\begin{block}{Exemple}
		Pour la chaîne $s = a_{1} b_{2} c_{3} a_{4} a_{5} d_{6} d_{7} b_{8}$, le \textit{fingerprint} de $[3,7]$ est $cad$.
	\end{block}
	
	Cet article s'intéresse au problème algorithmique du calcul de l'ensemble $\mathcal{F}$ dans $s$.	
\end{frame}

\begin{frame}
	\frametitle{Principes}
	
	\begin{enumerate}
		\item Construire l'arbre des suffixes de $s$
		\item À partir de l'arbre des suffixes, construire un arbre des participations
		\item À partir de l'arbre des participations, nommer tous les \textit{fingerprints} de $s$
	\end{enumerate}

\end{frame}

\begin{frame}
	\frametitle{Paramètres de l'algorithme : locations maximales}
	
	\begin{block}{Définition}
		On note $\mathcal{C}$ un ensemble de lettres de $\Sigma$. Une location maximale de $\mathcal{C}$ dans $s$ est un intervalle $[i,j]$ tel que :
		
		\begin{itemize}
			\item $\mathcal{C}_{s}(i,j) = \mathcal{C}$
			\item Si $i > 1$, $s_{i-1} \notin$ $\mathcal{C}_{s}(i,j)$
			\item Si $j < n$, $s_{j+1} \notin$ $\mathcal{C}_{s}(i,j)$
		\end{itemize}
		
		On note $\mathcal{L}$ l'ensemble des locations maximales de tous les \textit{fingerprints} de $\mathcal{F}$.
	\end{block}
	
	\begin{block}{Exemple}
		Pour la chaîne $s = a_{1} b_{2} c_{3} a_{4} a_{5} d_{6} d_{7} b_{8}$, la seule location maximale de $abc$ est $\langle1,5\rangle$.
	\end{block}
\end{frame}

\begin{frame}
	\frametitle{Paramètres de l'algorithme : copies}
	
	\begin{block}{Définition}
		Deux locations maximales $\langle i,j \rangle$ et $\langle k,l \rangle$ de $s$ sont des copies si $s_{i}..s_{j} = s_{k}..s_{l}$.\newline
		
		On note $\mathcal{L}_{\mathcal{C}}$ l'ensemble des classes d'équivalence.
	\end{block}
	
	\begin{block}{Exemple}
		Pour la chaîne $s = a_{1} b_{2} a_{3} d_{4} a_{5} b_{6} a_{7}$, $\langle1,3\rangle$ et $\langle5,7\rangle$ sont des copies.
	\end{block}
\end{frame}

\begin{frame}
	\frametitle{Travaux sur la complexité de l'algorithme}
	
	\begin{block}{Algorithme initial}
		A. Amir, A. Apostolico, G. M. Landau et G. Satta. \textit{Efficient text fingerprinting via parikh mapping}, 2003 :\newline
		
		$\mathcal{O}(n|\Sigma|$ $log$ $n$ $log$ $|\Sigma|)$
	\end{block}
	
	\begin{block}{Algorithme à améliorer}
		R. Kolpakov and M. Raffinot. \textit{New algorithms for text fingerprinting}, 2006 : \newline
		
		$\mathcal{O}(n$ $+$ $|\mathcal{L}|$ $log$ $|\Sigma|)$
	\end{block}
	
	\begin{block}{Algorithme introduit dans \textit{Faster Text Fingerprinting}}
		$\mathcal{O}(n$ $+$ $|\mathcal{L}_{\mathcal{C}}|)$ $log$ $|\Sigma|)$
	\end{block}

\end{frame}


\section{Participation Tree}

\section{Fingerprint Naming}

\section{Conclusion}

\end{document}